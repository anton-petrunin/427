\documentclass[oneside,a4paper]{article}
\usepackage{lectures}
\usepackage{xr}
\usepackage{ulem}
\externaldocument{all-lectures}

\begin{document}

\title{Extra credit problems}
\author{Math 427}
\date{}
\maketitle

%\textit{(solved)}\sout

\noi 0. Find a mistake or misprint in the book.
(The score depends on the type of mistake.)

\ 

\noi %\sout
{1.} Describe all the motions of the Manhattan plane.

\ 

\noi \sout{2.} \textit{(solved)}
Construct a metric space $\mathcal X$ and a distance preserving map $f\:\mathcal X\to \mathcal X$ that is not a motion of $\mathcal X$.

\ 

\noi %\sout
{3.} Note that the following quantity 
$$\tilde
\measuredangle ABC=\left[
\begin{aligned}
&\pi&&\text{if}&\measuredangle ABC&=\pi
\\
-&\measuredangle ABC&&\text{if}&\measuredangle ABC&<\pi
\end{aligned}
\right.$$
can serve as the angle measure; 
that is, the axioms hold if one changes everywhere $\measuredangle$ to $\tilde\measuredangle$.

\noi (a). Show that $\measuredangle$ and $\tilde\measuredangle$ are the only possible angle measures on the plane. 

\noi (b). Show that without Axiom IIIc, this is not longer true.


\ 

\noi %\sout
{4.} %\textit{(solved)}
Show that a composition of three reflections in the sides of a nondegenerate triangle does not have a fixed point.

\ 

\noi %\sout
{5.} %\textit{(solved)}
Consider $\triangle A B C$ with $D\in (A C)$ such that $(BD)\perp (AC)$, and points $N$ and $M$ such that $AN=DC$, $CM=AD$, $(AN)\perp(AB)$ and $(CM)\perp(BC)$.

Prove that $M$ and $N$ are equidistant form $B$.

\ 

\noi %\sout
{6.} %\textit{(solved)} 
Lines $\ell$ and $m$ are tangent to two circles of radiuses $r$ and $R$ on such
a way the circles are on one side of $\ell$ and on different sides of $m$. 
Let $A$ and $B$ be
tangential points of $\ell$ and $Q$ be the point of intersection $\ell$ and $m$. 
Show that
$$QA\cdot QB = R\cdot r.$$

\ 

\noi %\sout
{7.} %\textit{(solved)} 
Given a line segment with marked midpoint, make a ruler-only construction a line through a given point $P$ parallel to the line containing the segment.

\ 

\noi %\sout
{8.} Let $ABC$ be a nondegenerate triangele and $A'\in (BC)$, $B'\in (CA)$, $C'\z\in (AB)$ be the points such that 
\[2\cdot \angle AA'B\equiv 2\cdot \angle BB'C\equiv 2\cdot \angle CC'A\equiv \tfrac23{\cdot}\pi.\]
Show that the triangle formed by the lines $(BB')$, $(CC')$, and $(AA')$, is congruent to  $\triangle ABC$.

%\ 

%\noi %\sout
%{10.} Construct a triangle with the given perimeter, base, and the opposite angle
%(You can play with the java applet ``Triangle with given base, perimeter and angle.html'' on  \href{http://anton-petrunin.github.io/birkhoff/car/}{anton-petrunin.github.io/birkhoff/car/}.)

\ 

\noi %\sout
{9.} Two points $A$ and $B$ lie on one side of a line $\ell$. 
Two points $M$ and $N$ are chosen on $\ell$ such that $AM + BM$ is minimal and $AN = BN$. 
Show that points $A$, $B$, $M$ and $N$ lie on one circle.


\ 

\noi %\sout
{10.}
Let $\Gamma$ be a circle with the center $O$ and $A$ and $C$ be two different points on $\Gamma$. For any third point $P$ of the circle let $X$ and $Y$ be the midpoints of the segments $AP$ and $CP$. Finally, let $H$ be the orthocenter of the triangle $OXY$. Prove that the position of the point $H$ does not depend on the choice of $P$.

\ 

\noi %\sout
{11.}
Suppose $D$ and $E$ lie on the same side from $(AC)$, $(AE)\parallel (CD)$, and $AB=BC$.
Let $K$ be the intersection of the bisectors of the angles $EAB$ and $BCD$.
Prove that $(BK) \parallel (AE)$.

%\ 

%\noi %\sout
%{14.} Give a ruler-and-compass construction of a circle or a line which perpendicular to each of three given circles. 
%(You may assume any two of three given circles do not intersect. Play with the java applet ``Perpendicular to 3 circles.html'' on  \href{http://anton-petrunin.github.io/birkhoff/car/}{anton-petrunin.github.io/birkhoff/car/}.)


\ 

\noi %\sout
{12.} Give a ruler-and-compass construction of an inscribed quadrilateral with given sides.

\end{document}

\ 

\noi %\sout
{16.} %\textit{(solved)} 
Show that a neutral plane is Euclidean if and only if it has a rectangle.

\ 

\noi %\sout
{17.} Let $ABCDE$ be a regular right-angled pentagon in the hyperbolic plane;
that is, 
\[AB_h\z=BC_h=CD_h=DE_h=EA_h\] 
and 
\[\measuredangle_h ABC=\measuredangle_h BCD=\measuredangle_h CDE=\measuredangle_h DEA=\measuredangle_h EAB=\pm\tfrac\pi2.\]
Find its side $AB_h$.



