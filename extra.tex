\documentclass[oneside,a4paper]{article}
\usepackage{lectures}
\usepackage{xr}
\usepackage{ulem}
\externaldocument{all-lectures}

\begin{document}

\title{Extra credit problems}
\author{Math 427}
\date{}
\maketitle

\textit{}

\noi 1. Describe all the motions of the Manhattan plane.

\ 

\noi 2. Construct a metric space $\mathcal X$ and a distance preserving map $f\:\mathcal X\to \mathcal X$ which is not a motion of $\mathcal X$.

\ 

\noi 3. Note that the following quantity 
$$\tilde
\measuredangle ABC=\left[
\begin{aligned}
&\pi&&\text{if}&\measuredangle ABC&=\pi
\\
-&\measuredangle ABC&&\text{if}&\measuredangle ABC&<\pi
\end{aligned}
\right.$$
can serve as the angle measure; 
that is, the axioms hold if one changes everywhere $\measuredangle$ to $\tilde\measuredangle$.

\noi (a). Show that $\measuredangle$ and $\tilde\measuredangle$ are the only possible angle measures on the plane. 

\noi (b). Show that without Axiom IIc, this is not longer true.

\end{document}

\ 




\ 

\noi 4. Consider triangle $\triangle A B C$ with $D\in (A C)$ such that $(BD)\perp (AC)$, and points $N$ and $M$ such that $AN=DC$, $CM=AD$, $(AN)\perp(AB)$ and $(CM)\perp(BC)$.

Prove that $M$ and $N$ are equidistant form $B$. 

\ 

%\noi 5. Two points $A$ and $B$ lie on one side of line $\ell$. 
%Two points $M$ and $N$ are chosen on $\ell$ such that $AM + BM$ is minimal and $AN = BN$. 
%Show that points $A$, $B$, $M$ and $N$ lie on one circle.

%\ 

\noi 5.  Lines $\ell$ and $m$ are tangent to two circles of radii $r$ and $R$ on such
a way the circles are on one side of $\ell$ and on different sides of $m$. 
Let $A$ and $B$ be
tangential points of $\ell$ and $Q$ be the point of intersection $\ell$ and $m$. 
Show that
$$QA\cdot QB = R\cdot r.$$

\ 

\noi \sout{6.} \textit{(solved)} Given two parallel lines $\ell$ and $m$ and a point $P$, use only ruler to construct the line through $P$ parallel to $\ell$ and $m$.
(You can play with \href{https://dl.dropboxusercontent.com/u/1577084/m427/car/3d-parallel.html}{this java applet}.)

\ 

%\noi 7.  A hexagon $A B C D E F$ is inscribed in a circle and its main diagonals $[AD]$, $[BE]$ and $[CF]$ intersect at one point show that
%$$A B\cdot C D \cdot E F = BC\cdot DE \cdot F A.$$

%\ 

%\noi 8. Assume you have an instrument which makes possible to draw a circle or line through any given three points. 
%Show that it is impossible to construct the center of given circle using only this instrument.

%\ 

%\noi 10. Given two concentic circles construct their center using only ruler.
%(You can play with \href{http://dl.dropbox.com/u/1577084/m427/car/concetric_circles.html}{this java applet}.)

%\ 

%\noi 11. Show that any construction with only ruler can be done with a ``short ruler'';
%i.e. an instrument which makes possible to draw a line only through sufficiently close pair of points.

%\ 

%\noi 12. Give a ruler-and-compass construction of a circle or a line which perpendicular to each of three given circles. 
%(You may assume any two of three given circles do not intersect.) 

